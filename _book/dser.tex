\documentclass[]{book}
\usepackage{lmodern}
\usepackage{amssymb,amsmath}
\usepackage{ifxetex,ifluatex}
\usepackage{fixltx2e} % provides \textsubscript
\ifnum 0\ifxetex 1\fi\ifluatex 1\fi=0 % if pdftex
  \usepackage[T1]{fontenc}
  \usepackage[utf8]{inputenc}
\else % if luatex or xelatex
  \ifxetex
    \usepackage{mathspec}
  \else
    \usepackage{fontspec}
  \fi
  \defaultfontfeatures{Ligatures=TeX,Scale=MatchLowercase}
\fi
% use upquote if available, for straight quotes in verbatim environments
\IfFileExists{upquote.sty}{\usepackage{upquote}}{}
% use microtype if available
\IfFileExists{microtype.sty}{%
\usepackage{microtype}
\UseMicrotypeSet[protrusion]{basicmath} % disable protrusion for tt fonts
}{}
\usepackage[margin=1in]{geometry}
\usepackage{hyperref}
\hypersetup{unicode=true,
            pdftitle={Data Analysis for Educational Research in R},
            pdfauthor={Joshua M. Rosenberg},
            pdfborder={0 0 0},
            breaklinks=true}
\urlstyle{same}  % don't use monospace font for urls
\usepackage{natbib}
\bibliographystyle{apalike}
\usepackage{color}
\usepackage{fancyvrb}
\newcommand{\VerbBar}{|}
\newcommand{\VERB}{\Verb[commandchars=\\\{\}]}
\DefineVerbatimEnvironment{Highlighting}{Verbatim}{commandchars=\\\{\}}
% Add ',fontsize=\small' for more characters per line
\usepackage{framed}
\definecolor{shadecolor}{RGB}{248,248,248}
\newenvironment{Shaded}{\begin{snugshade}}{\end{snugshade}}
\newcommand{\KeywordTok}[1]{\textcolor[rgb]{0.13,0.29,0.53}{\textbf{#1}}}
\newcommand{\DataTypeTok}[1]{\textcolor[rgb]{0.13,0.29,0.53}{#1}}
\newcommand{\DecValTok}[1]{\textcolor[rgb]{0.00,0.00,0.81}{#1}}
\newcommand{\BaseNTok}[1]{\textcolor[rgb]{0.00,0.00,0.81}{#1}}
\newcommand{\FloatTok}[1]{\textcolor[rgb]{0.00,0.00,0.81}{#1}}
\newcommand{\ConstantTok}[1]{\textcolor[rgb]{0.00,0.00,0.00}{#1}}
\newcommand{\CharTok}[1]{\textcolor[rgb]{0.31,0.60,0.02}{#1}}
\newcommand{\SpecialCharTok}[1]{\textcolor[rgb]{0.00,0.00,0.00}{#1}}
\newcommand{\StringTok}[1]{\textcolor[rgb]{0.31,0.60,0.02}{#1}}
\newcommand{\VerbatimStringTok}[1]{\textcolor[rgb]{0.31,0.60,0.02}{#1}}
\newcommand{\SpecialStringTok}[1]{\textcolor[rgb]{0.31,0.60,0.02}{#1}}
\newcommand{\ImportTok}[1]{#1}
\newcommand{\CommentTok}[1]{\textcolor[rgb]{0.56,0.35,0.01}{\textit{#1}}}
\newcommand{\DocumentationTok}[1]{\textcolor[rgb]{0.56,0.35,0.01}{\textbf{\textit{#1}}}}
\newcommand{\AnnotationTok}[1]{\textcolor[rgb]{0.56,0.35,0.01}{\textbf{\textit{#1}}}}
\newcommand{\CommentVarTok}[1]{\textcolor[rgb]{0.56,0.35,0.01}{\textbf{\textit{#1}}}}
\newcommand{\OtherTok}[1]{\textcolor[rgb]{0.56,0.35,0.01}{#1}}
\newcommand{\FunctionTok}[1]{\textcolor[rgb]{0.00,0.00,0.00}{#1}}
\newcommand{\VariableTok}[1]{\textcolor[rgb]{0.00,0.00,0.00}{#1}}
\newcommand{\ControlFlowTok}[1]{\textcolor[rgb]{0.13,0.29,0.53}{\textbf{#1}}}
\newcommand{\OperatorTok}[1]{\textcolor[rgb]{0.81,0.36,0.00}{\textbf{#1}}}
\newcommand{\BuiltInTok}[1]{#1}
\newcommand{\ExtensionTok}[1]{#1}
\newcommand{\PreprocessorTok}[1]{\textcolor[rgb]{0.56,0.35,0.01}{\textit{#1}}}
\newcommand{\AttributeTok}[1]{\textcolor[rgb]{0.77,0.63,0.00}{#1}}
\newcommand{\RegionMarkerTok}[1]{#1}
\newcommand{\InformationTok}[1]{\textcolor[rgb]{0.56,0.35,0.01}{\textbf{\textit{#1}}}}
\newcommand{\WarningTok}[1]{\textcolor[rgb]{0.56,0.35,0.01}{\textbf{\textit{#1}}}}
\newcommand{\AlertTok}[1]{\textcolor[rgb]{0.94,0.16,0.16}{#1}}
\newcommand{\ErrorTok}[1]{\textcolor[rgb]{0.64,0.00,0.00}{\textbf{#1}}}
\newcommand{\NormalTok}[1]{#1}
\usepackage{longtable,booktabs}
\usepackage{graphicx,grffile}
\makeatletter
\def\maxwidth{\ifdim\Gin@nat@width>\linewidth\linewidth\else\Gin@nat@width\fi}
\def\maxheight{\ifdim\Gin@nat@height>\textheight\textheight\else\Gin@nat@height\fi}
\makeatother
% Scale images if necessary, so that they will not overflow the page
% margins by default, and it is still possible to overwrite the defaults
% using explicit options in \includegraphics[width, height, ...]{}
\setkeys{Gin}{width=\maxwidth,height=\maxheight,keepaspectratio}
\IfFileExists{parskip.sty}{%
\usepackage{parskip}
}{% else
\setlength{\parindent}{0pt}
\setlength{\parskip}{6pt plus 2pt minus 1pt}
}
\setlength{\emergencystretch}{3em}  % prevent overfull lines
\providecommand{\tightlist}{%
  \setlength{\itemsep}{0pt}\setlength{\parskip}{0pt}}
\setcounter{secnumdepth}{5}
% Redefines (sub)paragraphs to behave more like sections
\ifx\paragraph\undefined\else
\let\oldparagraph\paragraph
\renewcommand{\paragraph}[1]{\oldparagraph{#1}\mbox{}}
\fi
\ifx\subparagraph\undefined\else
\let\oldsubparagraph\subparagraph
\renewcommand{\subparagraph}[1]{\oldsubparagraph{#1}\mbox{}}
\fi

%%% Use protect on footnotes to avoid problems with footnotes in titles
\let\rmarkdownfootnote\footnote%
\def\footnote{\protect\rmarkdownfootnote}

%%% Change title format to be more compact
\usepackage{titling}

% Create subtitle command for use in maketitle
\newcommand{\subtitle}[1]{
  \posttitle{
    \begin{center}\large#1\end{center}
    }
}

\setlength{\droptitle}{-2em}
  \title{Data Analysis for Educational Research in R}
  \pretitle{\vspace{\droptitle}\centering\huge}
  \posttitle{\par}
  \author{Joshua M. Rosenberg}
  \preauthor{\centering\large\emph}
  \postauthor{\par}
  \predate{\centering\large\emph}
  \postdate{\par}
  \date{2017-07-30}

\usepackage{booktabs}

\usepackage{amsthm}
\newtheorem{theorem}{Theorem}[chapter]
\newtheorem{lemma}{Lemma}[chapter]
\theoremstyle{definition}
\newtheorem{definition}{Definition}[chapter]
\newtheorem{corollary}{Corollary}[chapter]
\newtheorem{proposition}{Proposition}[chapter]
\theoremstyle{definition}
\newtheorem{example}{Example}[chapter]
\theoremstyle{remark}
\newtheorem*{remark}{Remark}
\begin{document}
\maketitle

{
\setcounter{tocdepth}{1}
\tableofcontents
}
\chapter{Introduction}\label{introduction}

Educational research is hard to do (Berliner, 2002). This is because
many educational phenomena are part of a complex system, with multiple,
nested levels, and, well, people, many of them developing. Data analysis
in education reflects some of the challenges of educational research
writ large. In short, both educational research and analysis of
educational data is hard. The goal of this book is to share how to make
these difficulties less challenging using R, the open-source, free
programming language and software.

\chapter{Why a book on data science in educational
research}\label{why-a-book-on-data-science-in-educational-research}

There are at least three reasons why data analysis in educational
research is hard:

\begin{itemize}
\item
  Educational researchers have unique methods: an emphasis on
  multi-level models, networks, and measurement are just some examples.
\item
  Educational researchers face unique challenges: coming from myriad
  backgrounds, and working in fields with greater or lesser emphases on
  different aspects of data analysis.
\item
  Finally, there are training challenges. Educational research features
  some great methodologists: Many advances in the fields mentioned
  earlier in this session have been made by those working primarily in
  educational research. Nevertheless, few quantitative classes teach
  data analysis.
\end{itemize}

\chapter{Getting started and loading
data}\label{getting-started-and-loading-data}

What you want to do:

\begin{itemize}
\item
  Install R Studio and R
\item
  Load data saved in a Microsoft Excel spreadsheet (\texttt{.xlsx}),
  comma separated values file (\texttt{.csv}), SPSS file
  (\texttt{.sav}), or Google Sheet.
\end{itemize}

\section{Install R Studio and R}\label{install-r-studio-and-r}

First, you'll need to download the latest versions of R Studio and R.
Although we'll exclusively be using R Studio, R Studio needs to have R
installed, as well. You can find links here:

Download R Studio:
\url{https://www.rstudio.com/products/rstudio/download/\#download}

Download R: \url{https://cran.r-project.org/}

Don't worry; you won't mess anything up if you download (or even
install!) the wrong file. Once you've installed both, we can get to work
doing some data analysis.

\section{Loading data}\label{loading-data}

You might be thinking that an Excel file is the first that we would
load, but there happens to be a format which you can open and edit in
Excel that is even easier to use between Excel and R and among Excel, R,
as well as SPSS and other statistical software, like MPlus, and even
other programming languages, like Python. That format is CSV, or a
comma-separated-values file.

The CSV file is useful because you can open it with Excel and save Excel
files as CSV files. Additionally, and as its name indicates, a CSV file
is rows of a spreadsheet with the columns separated by commas, so you
can view it in a text editor, like TextEdit for Macintosh, as well. Not
surprisingly, Google Sheets easily converts CSV files into a Sheet, and
also easily saves Sheets as CSV files.

For these reasons, we start with reading CSV files.

\subsection{Loading CSV files}\label{loading-csv-files}

The easiest way to read a CSV file is with the function
\texttt{read\_csv()}. It is from a package, or an add-on to R, called
\texttt{readr}, but we are going to install \texttt{readr} as well as
other packages that work well together as part of a group of packages
named the \texttt{tidyverse}. To install all of the packages in the
tidyverse, use the following command:

\begin{Shaded}
\begin{Highlighting}[]
\KeywordTok{install.packages}\NormalTok{(}\StringTok{"tidyverse"}\NormalTok{)}
\end{Highlighting}
\end{Shaded}

You can also navigate to the Packages pane, and then click ``Install'',
which will work the same as the line of code above. Note, here there is
a way to install a package using code or part of the R Studio interface.
Usually, using code is a bit quicker, but sometimes (as we will see in a
moment) using the interface can be very useful and sometimes
complimentary to use of code.

We have now installed the tidyverse. We only have to install a package
once, but to use it, we have to load it each time we start a new R
session. We will discuss what an R session is later on; for now, know
that we have to load a package to use it. We do that with
\texttt{library()}:

\begin{Shaded}
\begin{Highlighting}[]
\KeywordTok{library}\NormalTok{(tidyverse)}
\end{Highlighting}
\end{Shaded}

And now we can load a file. We are going to call the data
\texttt{student\_responses}:

Since we loaded the data, we now want to look at it. We can just type
its name, and a summary of the data will print:

This was a minor task, but if you loaded a file and printed it, give
yourself a pat on the back. It is no joke to say that many times simply
being able to load a file into new software. We are now well on our way
to carrying out analysis of our data.

\subsection{Loading Excel files}\label{loading-excel-files}

We will now do the same with an Excel file. You might be thinking that
you can simply open the file in Excel and then save it as a CSV. This is
generally a good idea. At the same time, sometimes you may need to
directly read a file from Excel, and it is easy enough to do this.

The package that we use, \texttt{readxl}, is not a part of the
tidyverse, so we will have to install it first (remember, we only need
to do this once), and then load it using \texttt{library(readxl)}. Note
that the command to install \texttt{readxl} is grayed-out below: The
\texttt{\#} symbol before \texttt{install.packages("readxl")} indicates
that this line should be treated as a comment and not actually run, like
the lines of code that are not grayed-out. It is here just as a reminder
that the package needs to be installed if it is not already.

Once we have installed readxl, we have to load it (just like tidyverse):

\begin{Shaded}
\begin{Highlighting}[]
\CommentTok{# install.packages("readxl")}
\KeywordTok{library}\NormalTok{(readxl)}
\end{Highlighting}
\end{Shaded}

We can then use \texttt{read\_xlsx()} in the same way as
\texttt{read\_csv()}:

\begin{Shaded}
\begin{Highlighting}[]
\NormalTok{x <-}\StringTok{ }\KeywordTok{read_xls}\NormalTok{()}
\NormalTok{x <-}\StringTok{ }\KeywordTok{read_xlsx}\NormalTok{()}
\end{Highlighting}
\end{Shaded}

And we can print the data we loaded in the same way:

\section{Loading SAV files}\label{loading-sav-files}

The same factors that apply to reading Excel files apply to reading
\texttt{SAV} files (from SPSS). First, install the package
\texttt{haven}, load it, and the use the function \texttt{read\_sav()}:

\begin{Shaded}
\begin{Highlighting}[]
\CommentTok{# install.packages("haven")}
\KeywordTok{library}\NormalTok{(haven)}

\NormalTok{x <-}\StringTok{ }\KeywordTok{read_sav}\NormalTok{(}\StringTok{""}\NormalTok{)}
\end{Highlighting}
\end{Shaded}

\subsection{Google Sheets}\label{google-sheets}

Finally, it can sometimes be useful to load a file directly from Google
Sheets, and this can be done using the Google Sheets package.

\begin{Shaded}
\begin{Highlighting}[]
\CommentTok{# install.packages("googlesheets")}
\KeywordTok{library}\NormalTok{(googlesheets)}
\end{Highlighting}
\end{Shaded}

When you run the command below, a link to authenticate with your Google
account will open in your browser.

\begin{Shaded}
\begin{Highlighting}[]
\NormalTok{my_sheets <-}\StringTok{ }\KeywordTok{gs_ls}\NormalTok{()}
\end{Highlighting}
\end{Shaded}

You can then simply use the \texttt{gs\_title()} function in conjunction
with the \texttt{gs\_read()} function:

\begin{Shaded}
\begin{Highlighting}[]
\NormalTok{df <-}\StringTok{ }\KeywordTok{gs_title}\NormalTok{(}\StringTok{'title'}\NormalTok{)}
\NormalTok{df <-}\StringTok{ }\KeywordTok{gs_read}\NormalTok{(df)}
\end{Highlighting}
\end{Shaded}

\section{Conclusion}\label{conclusion}

Great job! You've made it through the first chapter. For more on reading
files, we will discuss how to use functions to read every file in a
folder (or, to write many different files to a folder).

\chapter{Processing and Tidying Data}\label{processing-and-tidying-data}

\section{What you want to do}\label{what-you-want-to-do}

\begin{itemize}
\tightlist
\item
  Create new variables
\item
  Select some cases
\item
  Join data
\end{itemize}

\chapter{Processing and Tidying
Data}\label{processing-and-tidying-data-1}

\section{What you want to do}\label{what-you-want-to-do-1}

\begin{itemize}
\tightlist
\item
  Go from wide form to long form
\item
  Go from long form to wide form
\item
  Aggregate data
\end{itemize}

\chapter{Data Visualization}\label{data-visualization}

We have finished a nice book.

\chapter{Linear Models (Regression and
ANOVA)}\label{linear-models-regression-and-anova}

Some \emph{significant} applications are demonstrated in this chapter.

\chapter{Linear Mixed Effects Models}\label{linear-mixed-effects-models}

\chapter{Social Network Analysis}\label{social-network-analysis}

\chapter{Factor Analysis}\label{factor-analysis}

\chapter{Latent Variable Models}\label{latent-variable-models}

\chapter{Natural Language Processing}\label{natural-language-processing}

\chapter{Cluster Analysis}\label{cluster-analysis}

\chapter{Reproducibility}\label{reproducibility}

\bibliography{packages.bib,book.bib}


\end{document}
